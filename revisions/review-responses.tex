%        File: revise.tex
%     Created: Wed Oct 27 02:00 PM 2018 P
% Last Change: Wed Oct 27 02:00 PM 2018 P
%

%
% Copyright 2007, 2008, 2009 Elsevier Ltd
%
% This file is part of the 'Elsarticle Bundle'.
% ---------------------------------------------
%
% It may be distributed under the conditions of the LaTeX Project Public
% License, either version 1.2 of this license or (at your option) any
% later version.  The latest version of this license is in
%    http://www.latex-project.org/lppl.txt
% and version 1.2 or later is part of all distributions of LaTeX
% version 1999/12/01 or later.
%
% The list of all files belonging to the 'Elsarticle Bundle' is
% given in the file `manifest.txt'.
%

% Template article for Elsevier's document class `elsarticle'
% with numbered style bibliographic references
% SP 2008/03/01
%
%
%
% $Id: elsarticle-template-num.tex 4 2009-10-24 08:22:58Z rishi $
%
%
%\documentclass[preprint,12pt]{elsarticle}
\documentclass[answers,11pt]{exam}

% \documentclass[preprint,review,12pt]{elsarticle}

% Use the options 1p,twocolumn; 3p; 3p,twocolumn; 5p; or 5p,twocolumn
% for a journal layout:
% \documentclass[final,1p,times]{elsarticle}
% \documentclass[final,1p,times,twocolumn]{elsarticle}
% \documentclass[final,3p,times]{elsarticle}
% \documentclass[final,3p,times,twocolumn]{elsarticle}
% \documentclass[final,5p,times]{elsarticle}
% \documentclass[final,5p,times,twocolumn]{elsarticle}

% if you use PostScript figures in your article
% use the graphics package for simple commands
% \usepackage{graphics}
% or use the graphicx package for more complicated commands
\usepackage{graphicx}
% or use the epsfig package if you prefer to use the old commands
% \usepackage{epsfig}

% The amssymb package provides various useful mathematical symbols
\usepackage{amssymb}
% The amsthm package provides extended theorem environments
% \usepackage{amsthm}
\usepackage{amsmath}

% The lineno packages adds line numbers. Start line numbering with
% \begin{linenumbers}, end it with \end{linenumbers}. Or switch it on
% for the whole article with \linenumbers after \end{frontmatter}.
\usepackage{lineno}

% I like to be in control
\usepackage{placeins}

% natbib.sty is loaded by default. However, natbib options can be
% provided with \biboptions{...} command. Following options are
% valid:

%   round  -  round parentheses are used (default)
%   square -  square brackets are used   [option]
%   curly  -  curly braces are used      {option}
%   angle  -  angle brackets are used    <option>
%   semicolon  -  multiple citations separated by semi-colon
%   colon  - same as semicolon, an earlier confusion
%   comma  -  separated by comma
%   numbers-  selects numerical citations
%   super  -  numerical citations as superscripts
%   sort   -  sorts multiple citations according to order in ref. list
%   sort&compress   -  like sort, but also compresses numerical citations
%   compress - compresses without sorting
%
% \biboptions{comma,round}

% \biboptions{}


% Katy Huff addtions
\usepackage{xspace}
\usepackage{color}

\usepackage{multirow}
\usepackage[hyphens]{url}


\usepackage[acronym,toc]{glossaries}
\include{acros}

\makeglossaries

%\journal{Annals of Nuclear Energy}

\begin{document}

%\begin{frontmatter}

% Title, authors and addresses

% use the tnoteref command within \title for footnotes;
% use the tnotetext command for the associated footnote;
% use the fnref command within \author or \address for footnotes;
% use the fntext command for the associated footnote;
% use the corref command within \author for corresponding author footnotes;
% use the cortext command for the associated footnote;
% use the ead command for the email address,
% and the form \ead[url] for the home page:
%
% \title{Title\tnoteref{label1}}
% \tnotetext[label1]{}
% \author{Name\corref{cor1}\fnref{label2}}
% \ead{email address}
% \ead[url]{home page}
% \fntext[label2]{}
% \cortext[cor1]{}
% \address{Address\fnref{label3}}
% \fntext[label3]{}


\title{Standardized Verification of the Cyclus Fuel Cycle Simulator\\
        \large Response to Review Comments}
\author{Jin Whan Bae, Joshua L. Peterson-Droogh, Kathryn D. Huff}

% use optional labels to link authors explicitly to addresses:
% \author[label1,label2]{<author name>}
% \address[label1]{<address>}
% \address[label2]{<address>}


%\author[uiuc]{Kathryn Huff}
%        \ead{kdhuff@illinois.edu}
%  \address[uiuc]{Department of Nuclear, Plasma, and Radiological Engineering,
%        118 Talbot Laboratory, MC 234, Universicy of Illinois at
%        Urbana-Champaign, Urbana, IL 61801}
%
% \end{frontmatter}
\maketitle
\section*{Review General Response}
We would like to thank the reviewer for their prompt and detailed assessment of 
this paper. Your comments have resulted in changes which certainly improved the 
paper.


\begin{questions}
%---------------------------------------------------------------------
\section*{Reviewer 1}

        %---------------------------------------------------------------------
        \question This is a valuable contribution because it adds results from 
        Cyclus to an existing comparison.  Cyclus has growing importance in the 
        fuel cycle community and such comparisons are important for 
        establishing its role and credibility.  On the whole, this technical 
        note is well put together and makes a strong case for Cyclus in the 
        context of this particular scenario/transition.

        \begin{solution}
        <++>
        \end{solution}

        %---------------------------------------------------------------------
        \question line 5: the word ``validation'' is used here, although 
        everywhere else is described as ``verification''.  Strictly speaking, 
        it may be neither, but (a) it is closer to verification and (b) a 
        single term should be used throughout.

        \begin{solution}
        <++>
        \end{solution}

        %---------------------------------------------------------------------
        \question line 11: ``excel'' should be capitalized and may need a 
        trademark symbol

        \begin{solution}
        <++>
        \end{solution}

        %---------------------------------------------------------------------
        \question line 27: strike ``the Institutions.'' (grammar) from the end 
        of the sentence.

        \begin{solution}
        <++>
        \end{solution}

        %---------------------------------------------------------------------
        \question lines 37/38: the description of file type and processing 
        tools seems like unnecessary detail.  Maybe it is related to the 
        statement about all the files being available on Zenodo, and thus 
        openly available.  If so, this point could be made more strongly, 
        possibly leading with the pointer to the open files/data, and then 
        describing its contents.

        \begin{solution}
        <++>
        \end{solution}

        %---------------------------------------------------------------------
        \question line 44: ``small edits in the source code'' - it may be worth 
        mentioning here, already, that these are changes are only to the 
        reactor archetype (even if more generic terminology is used)

        \begin{solution}
        <++>
        \end{solution}

        %---------------------------------------------------------------------
        \question lines 60-63: It is not clear why this was necessary. 
        Non-integer batches can simulated using the Cycamore reactor archetype 
        by loading a core with objects that are smaller than a batch (but 
        bigger than a real assembly for performance reasons).  In this problem, 
        removing 2 ``assemblies'' each cycle from a core loaded with 9 
        ``assemblies'' would achieve a 4.5 batch scheme.  If there was another 
        reason for choosing this modeling different, that should be mentioned.

        \begin{solution}
        <++>
        \end{solution}

        %---------------------------------------------------------------------
        \question end of section 3: There is merit in showing some summary of 
        the input used in Cyclus to model this problem.  At the very least, a 
        list of archetypes that were used and possibly a visualization of the 
        network that those archetypes create.  For each archetype, it may be 
        useful to comment on how straightforward the modeling choices were to 
        implement this specific scenario in the framework of the available 
        Cycamore archetypes.  You've given some detail for the reactor 
        archetype - perhaps the others were simpler, and don't need a complete 
        description, but there are often details in configuring the prototypes 
        that can be illustrative to others.  Recognizing the balance of keeping 
        this technical note brief, there may be room for a little more 
        description of the Cyclus scenario beyond the reactor configuration.

        \begin{solution}
        <++>
        \end{solution}

        %---------------------------------------------------------------------
        \question lines 79-81: Although it's not clear which Institution 
        archetype was used (see above), in the simplest case (DeployInst), the 
        reactor deployment information is input data and not the result of a 
        calculation.  The commentary around Figure 1 could mention that this is 
        a comparison of input data, showing that the scenarios are defined 
        equivalently.  It is an important comparison, but not really to be 
        interpreted as ``results''.

        \begin{solution}
        <++>
        \end{solution}

        %---------------------------------------------------------------------

        \question line 88: It is not clear what Figure 4 is showing.  (1) An 
        equation may be useful here to more clearly indicate how the 
        normalization works. It is not clear why this kind of normalization 
        doesn't give a uniform value of 1 if it's (delta fuel loading)/(delta 
        core mass).  At the very least additional commentary should describe 
        why/how ``This shows that the differences....''  (2) the figure's 
        y-axis is labeled for the normalized quantity, but not for the SFR 
        deployments - is that shown in ``number of reactors''?

        \begin{solution}
        <++>
        \end{solution}

        %---------------------------------------------------------------------

        \question line 93: ``and converges'' - what is converging to what?  
        Does it actually converge, or does the baseline become so small that 
        the oscillations can't be distinguished?

        \begin{solution}
        <++>
        \end{solution}

        %---------------------------------------------------------------------

        \question Figure 6: At least visually, the mean behavior over the 
        oscillations does not appear consistent with the reference.  Is this 
        the case? If so, comment.  If not, describe why it is not.

        \begin{solution}
        <++>
        \end{solution}

        %---------------------------------------------------------------------

\end{questions}
\bibliographystyle{unsrt}
\bibliography{review-responses}
\end{document}

  %
  % End of file `elsarticle-template-num.tex'.
