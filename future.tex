\section{Future Nuclear Projections}

The future of nuclear energy in EU nations is organized
in the table by the World Nuclear Association \cite{world_nuclear_association_nuclear_2017}.
It is assumed that all the planned constructions are completed in their expected date without delay
or failure. Also, the newly constructed nuclear power plants are assumed to have a lifetime of 60 years.

\Cref{tab:eu_deployment} lists the reactors that are currently under construction or planned.

 
\begin{table}[h]
	\centering
	\caption {Power Reactors under construction and planned \cite{world_nuclear_association_nuclear_2017}}
	\label{tab:eu_deployment}
	\scalebox{0.70}{
	\begin{tabular}{|c|c|c|c|c|}
		\hline
		Exp. Operational & Country & Reactor & Type & Gross MWe\\
		\hline
		2018 & Slovakia  & Mochovce 3 & PWR & 440\\
		2018 & Slovakia & Mochovce 4 & PWR & 440 \\
		2018 & France & Flamanville 3 & PWR & 1600 \\
		2018 & Finland & Olkilouto 3 & PWR & 1720 \\		
		2019 & Romania & Cernavoda 3 & PHWR & 720 \\
		2020 & Romania & Cernavoda 4 & PHWR & 720 \\
		2024 & Finland & Hanhikivi & VVER1200 & 1200 \\
		2024 & Hungary & Paks 5 & VVER1200 & 1200 \\
		2025 & Hungary & Paks 6 & VVER1200 & 1200 \\
		2025 & Bulgaria & Kozloduy 7 & AP1000? & 950 \\
		2026 & UK & Hinkley Point C1 & EPR & 1670 \\
		2027 & UK & Hinkley Point C2 & EPR & 1670 \\
		2029 & Poland & Choczewo? & N/A & 3000 \\
		2035 & Poland & East? & N/A & 3000 \\
		2035 & Czech Rep & Dukovany 5 & ? & 1200 \\
		2035 & Czech Rep & Temelin 3 & AP1000? & 1200 \\
		2040 & Czech Rep & Temelin 4 & AP1000? & 1200 \\
		\hline
	\end{tabular}
	}
\end{table}


For each EU nation, the growth trajectory is categorized from
"Aggressive Growth" to "Aggressive Shutdown". Aggressive growth is
characterized by a rigorous expansion of nuclear power while 
Aggressive Shutdown is characterized as a transition to rapidly
de-nuclearize the nation's electric grid. A nation's growth trajectory is
categorized into five spectra:

\begin{itemize}
	\item Aggressive Growth
	\item Modest Growth 
	\item Maintenance
	\item Modest Reduction
	\item Aggressive Reduction
\end{itemize}

The growth trajectory and specific plan of each nation in the EU 
is listed in \cref{tab:eu_growth}.

\begin{table}[h]
	\centering
		\begin{tabular}{|c|c|c|}
			\hline
			Nation & Growth Trajectory & Specific Plan \\
			\hline
			France & Modest Reduction & Shutdown nuclear plants if they reach end of lifetime. No new construction.\\
			Germany & Aggressive Reduction & Close all nuclear reactors by 2022.\\
			Czech Republic & Modest Growth & Additional 2,400 MWe (AP1000s) by 2035.\\
			UK & Aggressive Growth & 13 units (17,900 MWe) by 2030.\\
			Belgium & Aggressive Reduction & All shutdown 2025.\\
			Sweden & Aggressive Reduction & All shutdown 2050.\\
			Finland & Modest Growth & Additional EPR in 2018, VVER in 2024.\\
			Bulgaria & Modest Growth & Additional AP1000 (1,000 MWe) construction in 2035. \\
			Poland & Aggressive Growth & Additional 6,000 MWe by 2035.\\
			Romania & Modest Growth & Additional 1,440 MWe by 2020. \\
			Hungary & Modest Growth & Additional 2,400 MWe (VVER-1200) by 2025. \\ 
			Spain & Maintenance & No plans to expand or early shutdown. \\
			Italy & Maintenance  & No plans to expand or early shutdown. \\
			\hline
		\end{tabular}
	\caption {Future Nuclear Programs of EU Nations \cite{world_nuclear_association_nuclear_2017}}
	\label{tab:eu_growth}
\end{table}

