
\section{Introduction}
Fuel cycle simulators guide and inform \gls{NFC} research directions and policy choices.
Various institutions have developed fuel cycle simulators targeted at their unique needs,
using various methods and structures
to simulate material flow in the \gls{NFC}.
Algorithmic differences make
validation studies necessary to establish
confidence in software capabilities and
agreement amongst analysis tools.

A previous verification study \cite{feng_standardized_2016} compared
four well-known \gls{NFC} simulators
DYMOND \cite{yacout_modeling_2005},
VISION \cite{jacobson_verifiable_2010},
ORION \cite{gregg_analysis_2012}, and
MARKAL \cite{shay_epa_2006}. The results from each simulation tool
were compared to a set of `model solutions' that were generated
from an excel worksheet for different metrics (e.g. fuel loading in reactor,
\gls{UNF} inventory) in a transition scenario, and showed excellent agreement
with the result from the Feng et. al \cite{feng_standardized_2016}.

The present work benchmarks the \Cyclus nuclear fuel cycle simulator 
\cite{huff_fundamental_2016}
against the verification study results from Feng et. al 
\cite{feng_standardized_2016}.
Using the scenario definitions established in that validation study,
this work uses \Cyclus and its additional modules library, \Cycamore, to simulate the test case: a transition scenario from an
open fuel cycle to an advanced fuel cycle with
reprocessing. Comparison between these results and the Feng et. al `model solutions'
show excellent agreement.


\subsection{\Cyclus}

\Cyclus is an \emph{agent-based} fuel cycle simulation framework 
\cite{huff_fundamental_2016}, meaning 
that each reactor, reprocessing plant, fuel fabrication plant, and other fuel cycle
facility is modeled as a discrete entity.
A \Cyclus simulation contains prototypes, fuel cycle facilities with
pre-defined parameters, that are deployed as \texttt{Facility} agents.
\texttt{Institution} agents in \Cyclus deploy or decommission \texttt{Facility} 
agents which \texttt{Region} agents, in turn, manage the \texttt{Institution}s.
The \Cycamore library contains customized \texttt{Facility}, 
\texttt{Institution}, and \texttt{Region} models which vary in complexity and purpose \cite{huff_extensions_2014}.
The \Cycamore \texttt{DeployInst} is used as the \texttt{Institution} archetype for this 
 work, since it deploys facilities at user-defined timesteps.
