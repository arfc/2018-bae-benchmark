
\section{Introduction}
Fuel cycle simulators guide and inform \gls{NFC} research directions and policy choices.
Various institutions have developed fuel cycle simulators targeted at their unique needs,
using different methods and different structures
to simulate the material flow in the \gls{NFC}.
Algorithmic differences make
validation studies necessary to establish
confidence in software capabilities and
agreement amongst analysis tools.

A previous verification study \cite{feng_standardized_2016} compared
four well-known \gls{NFC} simulation codes
DYMOND \cite{yacout_modeling_2005},
VISION \cite{jacobson_verifiable_2010},
ORION \cite{gregg_analysis_2012}, and
MARKAL \cite{shay_epa_2006}. The results from each code
were compared to a set of `model solutions' that were generated
from an excel worksheet for different metrics (e.g. fuel loading in reactor,
\gls{UNF} inventory) in a transition scenario, and showed excellent agreement
with the result from the Feng et al \cite{feng_standardized_2016}.


This study benchmarks \Cyclus' results
against that of other well-known codes, such as
DYMOND \cite{yacout_modeling_2005},
VISION \cite{jacobson_verifiable_2010},
ORION \cite{gregg_analysis_2012}, and
MARKAL \cite{shay_epa_2006}. We take the input
parameters and results from a validation study
\cite{feng_standardized_2016} already done for the
mentioned tools for a transition scenario from an
open fuel cycle to an advanced fuel cycle with
reprocessing. In the benchmark, they compare the `model solutions'
generated from an excel worksheet
to the results from each code, and the results show
excellent agreement.


\subsection{\Cyclus}

\Cyclus is an agent-based fuel cycle simulation framework 
\cite{huff_fundamental_2016}, meaning 
that each reactor, reprocessing plant, fuel fabrication plant, and other fuel cycle
facility is modeled as a discrete entity.
A \Cyclus simulation contains prototypes, fuel cycle facilities with
pre-defined parameters, that are deployed as \texttt{facility} agents.
\texttt{Institution} and \texttt{Region} agents manage the \texttt{Facility} agents.
A \texttt{region} agent holds a set of \texttt{Institution}s.
An \texttt{Institution} agent can deploy or decommission \texttt{Facility} agents.
Several versions of \texttt{Institution}
and \texttt{Region} exist, varying in complexity and functions \cite{huff_extensions_2014}.
 \texttt{DeployInst} is used as the \texttt{Institution} archetype for this work, where it
deploys facilities at user-defined timesteps.
