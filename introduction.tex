\section{Abstract}
Numerous nuclear fuel cycle system modeling codes
have been developed to perform fuel cycle transition
analyses from a once-through cycle to an advanced
fuel cycle. Verification studies compare different
fuel cycle analysis tools against each other to
test agreement and identify sources of difference.
This paper benchmarks \Cyclus, the agent-based,
open-source fuel cycle simulation code, against
a verification study \cite{feng_standardized_2016} with
DYMOND \cite{yacout_2005_modeling},
VISION \cite{jacobson_2010_verifiable},
ORION \cite{gregg_2012_analysis}, and
MARKAL \cite{shay_2006_epa}. The study reveals
that \Cyclus' results match the spreadsheet results
very closely, with a minor difference with regard to
reprocessing \gls{UNF}. This difference is most
likely caused by two factors:
[GOOD WAY TO EXPLAIN]

\section{Introduction}
Fuel cycle simulators act as an important tool to
aid decision in policy and fuel cycle strategies.
To meet this need from various institutions, a
multitude of fuel cycle simulators were developed,
using different methods and different structures
to simulate the material flow in the nuclear fuel cycle.
The difference in the algorithm of fuel cycle analysis
codes combined with a small user community make
validation studies necessary to gain
confidence of the capability of the code as well as its
agreement with other analysis codes.

This study is done to benchmark \Cyclus' results
against that of other well-known codes, such as
DYMOND \cite{yacout_2005_modeling},
VISION \cite{jacobson_2010_verifiable},
ORION \cite{gregg_2012_analysis}, and
MARKAL \cite{shay_2006_epa}. We take the input
parameters and results from a validation study
\cite{feng_standardized_2016} already done for the
mentioned tools for a transition scenario from an
open fuel cycle to an advanced fuel cycle with
reprocessing. In the paper, the `model solutions'
generated from an excel worksheet are compared
with each code results, and the results show
excellent agreement.


\subsection{\Cyclus}

\Cyclus is an agent-based fuel cycle simulation framework 
\cite{huff_fundamental_2016}, which means 
that each reactor, reprocessing plant, and fuel fabrication plant is modeled as an agent.
A \Cyclus simulation contains prototypes, which are fuel cycle facilities with
pre-defined parameters, that are deployed in the simulation as \texttt{facility} agents.
Encapsulating the \texttt{facility} agents are the \texttt{Institution} and \texttt{Region}.
A \texttt{Region} agent holds a set of \texttt{Institution}s.
An \texttt{Institution} agent can deploy or decommission \texttt{facility} agents.
The \texttt{Institution} agent is part of a \texttt{Region} agent,
which can contain multiple \texttt{Institution} agents. Several versions of \texttt{Institution}
and \texttt{Region} exist, varying in complexity and functions \cite{huff_extensions_2014}.
 \texttt{DeployInst} is used as the institution archetype for this work, where the institution
deploys agents at user-defined timesteps.

At each timestep (one month),
agents make requests for materials or bid to supply them and exchange
with one another. A market-like mechanism called the dynamic resource exchange
\cite{gidden_agent-based_2015} governs the exchanges.
Each material resource has a quantity, composition, name, and a unique identifier
for output analysis. The timestep execution in \Cyclus follows 
\texttt{Build, Tick, \gls{DRE}, Tock, and Decommission}, as illustrated in
figure \ref{fig:time}. The \texttt{Tick}, and \texttt{Tock} phases are for
each agent to perform actions, such as transmutation, separation, generation,
of materials before and after the market exchange phase. 

\begin{figure}[h]
\centering
\scalebox{0.7}{
\begin{tikzpicture}[node distance=1.5cm]
\node (Build) [process] {Build (kernel)};
\node (Tick) [process, below of=Build] {Tick (agent)};
\node (DRE) [process, below of=Tick]{Dynamic Resource Exchange (kernel) };
\node (Tock) [process, below of=DRE]{Tock (agent)};
\node (Decom) [process, below of=Tock] {Decommission (kernel)};

\draw [arrow] (Build) -- (Tick); 
\draw [arrow] (Tick) -- (DRE);
\draw [arrow] (DRE) -- (Tock);
\draw [arrow] (Tock) -- (Decom);
\end{tikzpicture}
}
\caption{\Cyclus timestep execution steps.}
\label{fig:time}
\end{figure}

The modularity of \Cyclus allows a low barrier of
entry for developers, since developers can create an
archetype (e.g. Reactor module, Reprocessing module)
without extensive knowledge of the \Cyclus framework.
