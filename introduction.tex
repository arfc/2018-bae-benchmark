\section{Abstract}
Numerous nuclear fuel cycle system modeling codes
have been developed to perform fuel cycle transition
analyses from a once-through cycle to an advanced
fuel cycle. Verification studies compare different
fuel cycle analysis tools against each other to
test agreement and identify sources of difference.
This paper benchmarks \Cyclus, the agent-based,
open-source fuel cycle simulation code, against
a verification study \cite{feng_standardized_2016} with
DYMOND \cite{yacout_2005_modeling},
VISION \cite{jacobson_2010_verifiable},
ORION \cite{gregg_2012_analysis}, and
MARKAL \cite{shay_2006_epa}. The study reveals
that \Cyclus' results match the spreadsheet results
very closely, with a minor difference with regard to
reprocessing \gls{UNF}. This difference is most
likely caused by two factors. First, the timestep execution of \Cyclus causes delays in material
flow, since a material can move through one
facility at a timestep. Second, the buffers in
\Cyclus facilities causes difficulties when
analyzing time-dependent material flow.
However, the differences can be lessened with
iterations to the input file and possibly the
source code. Since \Cyclus is a modular code,
a facility module can be modified and used in
the simulation without changing the entire framework.




\section{Introduction}
Fuel cycle simulators act as an important tool to
aid decision in policy and fuel cycle strategies.
To meet this need from various institutions, a
multitude of fuel cycle simulators were developed,
using different methods and different structures
to simulate the material flow in the nuclear fuel cycle.
The difference in the algorithm of fuel cycle analysis
codes combined with a small user community make
validation studies necessary to gain
confidence of the capability of the code as well as its
agreement with other analysis codes.

This study is done to benchmark \Cyclus' results
against that of other well-known codes, such as
DYMOND \cite{yacout_2005_modeling},
VISION \cite{jacobson_2010_verifiable},
ORION \cite{gregg_2012_analysis}, and
MARKAL \cite{shay_2006_epa}. We take the input
parameters and results from a validation study
\cite{feng_standardized_2016} already done for the
mentioned tools for a transition scenario from an
open fuel cycle to an advanced fuel cycle with
reprocessing.


\subsection{\Cyclus}

\Cyclus is an agent-based fuel cycle simulation framework 
\cite{huff_fundamental_2016}, which means 
that each reactor, reprocessing plant, and fuel fabrication plant is modeled as an agent.
A \Cyclus simulation contains prototypes, which are fuel cycle facilities with
pre-defined parameters, that are deployed in the simulation as \texttt{facility} agents.
Encapsulating the \texttt{facility} agents are the \texttt{Institution} and \texttt{Region}.
A \texttt{Region} agent holds a set of \texttt{Institution}s.
An \texttt{Institution} agent can deploy or decommission \texttt{facility} agents.
The \texttt{Institution} agent is part of a \texttt{Region} agent,
which can contain multiple \texttt{Institution} agents. Several versions of \texttt{Institution}
and \texttt{Region} exist, varying in complexity and functions \cite{huff_extensions_2014}.
 \texttt{DeployInst} is used as the institution archetype for this work, where the institution
deploys agents at user-defined timesteps.

At each timestep (one month),
agents make requests for materials or bid to supply them and exchange
with one another. A market-like mechanism called the dynamic resource exchange
\cite{gidden_agent-based_2015} governs the exchanges.
Each material resource has a quantity, composition, name, and a unique identifier
for output analysis.

The modularity of \Cyclus allows a low barrier of
entry for developers, since developers can create an
archetype (e.g. Reactor module, Reprocessing module)
without extensive knowledge of the \Cyclus framework.


\section{Methodology}

The validation paper \cite{feng_standardized_2016}
has comprehensive simulation parameters that allows
us to reproduce the transition scenario in \Cyclus.
In this
study, we used the \Cycamore library to model
all fuel cycle facilities. \Cycamore libraries contain
simple fuel cycle facility models. For example,
the Reactor module does depletion calculations through
user-defined recipes.

\Cyclus' default timestep is a month. However, for this
study, since the verification paper mentions annual
rates of material flow or processing, we change the \Cyclus
simulation timestep to a year. This is done easily
by changing the simulation parameter \texttt{dt} in the input file.

\Cyclus outputs files in either \texttt{.sqlite} or
\texttt{.h5} format. In this study, we used the 
\texttt{.sqlite} format and analyzed the output file
using a python script.

The input file and analysis procedures are all in
[zenodo].


