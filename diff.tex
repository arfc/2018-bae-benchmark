
\section{Notable Modeling Differences in \Cyclus}

Notable algorithmic differences between \Cyclus and the fuel cycle analysis 
tools used in the benchmark \cite{feng_standardized_2016} are discussed here.

\Cyclus handles time, materials, and facilities discretely, and has a 1 month 
default time step,
The verification study solutions are evaluated with 1-year time steps, so cumulative and annual averages
were the basis for code-to-code comparison in this work.
For example, decommissioning of
facilities occurs at the end of a timestep, and building of facilities
occurs at the beginning of a timestep.

The \Cycamore recipe reactor depletes half of its core when decommissioned,
whereas the software tools in the benchmark \cite{feng_standardized_2016} deplete all their reactors' fuel when decommissioned.
This causes a discrepancy in the \gls{TRU} inventory. For this study, we 
the \Cycamore source code to deplete all its assemblies to the depleted recipe.
Also, the \Cycamore recipe reactor treats each batch (and assembly) as a discrete
material, while some tools assume continuous fuel discharge. This produces
differences in the results because the batches in the benchmark \cite{feng_standardized_2016} are in time-averaged values.

The benchmark unrealistically specifies fractional numbers of batches (4.5 and 
3.96 for \glspl{LWR} and \glspl{SFR} respectively) to 
accomodate software that operates on integer year timescales. \Cyclus is 
capable of more refined time resolution and accepts only integer 
numbers of batches, to better match the real world. Accordingly, the \gls{LWR} batch size, number of batches, and cycle time 
were adjusted to remain true to the benchmark specified core size. We rounded
up the \gls{SFR} batch number, while the batch size and cycle time are kept constant.
This increases the core size by $1.08 \%$, which is negligible, but will be
discussed in the results section.
The differences are listed in table \ref{tab:diff}.

\begin{table}[h]
    \centering
    \caption{Difference in Batch number and core size}
\begin{tabularx}{\textwidth}{bss}
        \hline
        \textbf{Category} & \textbf{Model Solution \cite{feng_standardized_2016}} & \textbf{\Cyclus} \\
        \hline
        LWR Batches & 4.5 & 3 \\
        LWR Batch size [tHM] & 19.91 & 29.86 \\
        LWR Core size [tHM] & 89.59 & 89.59 \\
        LWR Cycle time & 1 year & 1.5 years \\
        SFR Batches & 3.96 & 4 \\
        SFR Batch size [tHM] & 3.95 & 3.95 \\
        SFR Core size [tHM] & 15.63 & 15.8 \\
        \hline
        \end{tabularx}
        \label{tab:diff}
\end{table}

While any of these differences can be resolved by changing the \Cycamore
archetype source code, none require a change to the \Cyclus framework. Notably, 
        the only change made was the \Cycamore reactor
depletion behavior at decommission due to its large impact on plutonium inventory.
The goal of this
study is to show current \Cyclus agreement with other simulators and identify
differences, not to alter \Cyclus to match the other tools.
