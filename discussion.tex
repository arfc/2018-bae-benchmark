\section{Discussion}
This work demonstrated that, given infinite
reprocessing and \gls{MOX} fabrication capacities,
France can transition into a fully \gls{SFR} fleet
with installed capacity of 60,000 MWe by 2076.
The initial fuel demand is filled by \gls{MOX} from
reprocessed \gls{SNF}, which later on
is met by \gls{MOX} created from recycled \gls{MOX}.

Since most \gls{EU} nations do not have an operating \gls{SNF}
repository or a management plan, they have a strong incentive
to send all their \gls{SNF} to France. France has a financial
incentive (compensation from other nations) to take this fuel,
and since reuse of spent fuel from
other nations will allow France to meet their MOX demand
without new construction of \glspl{LWR}.

Though complex political and economic factors have not been
addressed, and various assumptions were made for this scenario,
this option may hold value for the \gls{EU} as a nuclear community,
and for France to advance into a closed fuel cycle.
