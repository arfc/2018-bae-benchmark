\section{Discussion}
This work demonstrated that, given infinite
reprocessing and \gls{MOX} fabrication capacities,
France, by receiving \gls{UNF} from other \gls{EU} nations,
can transition into a full \gls{SFR} fleet
with installed capacity of 60,000 MWe by 2076.
The initial fuel demand is filled by \gls{MOX} from
reprocessed \gls{UNF}, which later on
will be met by \gls{MOX} created from recycled \gls{MOX}.

Since most \gls{EU} nations do not have an operating \gls{UNF}
repository or a management plan, they have a strong incentive
to send all their \gls{UNF} to France. Especially, the nations
with aggressive nuclear reduction can phase out nuclear
without constructing a High Level Waste repository. France has a financial
incentive to take this fuel, since reuse of used fuel from
other nations will allow France to meet their MOX demand
without new construction of \glspl{LWR}.

Though complex political and economic factors have not been
addressed, and various assumptions were made for this scenario,
this option may hold value for the \gls{EU} as a nuclear community,
and for France to advance into a closed fuel cycle.
