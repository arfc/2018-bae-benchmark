\section{Methodology}

Feng et al. comprehensively defines simulation parameters
sufficient to reproduce the transition scenario in \Cyclus.
In this study, we used the \Cycamore \cite{huff_fundamental_2016}
 archetype library to model
all fuel cycle facilities. \Cycamore libraries contain
simple fuel cycle facility models. 

The \Cyclus input file for this simulation and analysis procedures are all
openly available in \cite{bae_arfc/transition-scenarios:_2018}.
\Cyclus results are output in either \texttt{.sqlite} or
\texttt{.h5} format. In this study, we used the
\texttt{.sqlite} format and analyzed the results
using a python script. We then compared the post-processed
output data to the results with the
model solution from the verification study \cite{feng_standardized_2016}.

The analysis and benchmark were performed iteratively,
where we improve the original result by communicating
with the authors of the benchmark. 
We analyzed the reasons for the differences from the original
result. Only one adjustment to the default \Cycamore reactor depletion behavior 
was needed, regarding fuel handling upon reactor decommissioning.
Otherwise, neither the \Cyclus frameework nor the facility behavior algorithms 
in \Cycamore required adjustment to perform the benchmark. All model heuristics 
contributing to minor deviations from the benchmark are explained in detail.


This simulation used seven commonly used \Cycamore \texttt{Facility} archetypes 
(Source, Enrichment, Reactor, Separations, Mixer, Storage, and Sink), two 
\texttt{Insitution} archetypes (NullInst and DeployInst), and one 
\texttt{Region} archetype (NullRegion). 
The DeployInst was used to deploy Reactor facilities on 
schedule while NullInst managed all other fuel cycle facilities. All resided in 
the single, simple, NullRegion.

Most of these facility archetypes were easily configured into agent prototypes 
using only a few variables. The Source was used to model a mine of natural 
uranium, while Sink facility protypes accepted waste streams of various kind 
(reprocessing waste, tails, etc.). The Enrichment facility archetype was 
similarly used to model uranium enrichment and is simple to configure. The 
Storage facility archetype was configured to model cooling times for the two 
types of fuel. 

More complex archetypee, the Separations and Mixer factility archetypes were 
configured to separate used fuel and fabricate it into new SFR fuel, conserving 
mass and modeling processing times. Additionally, the Reactor archetype was 
used for both reactor types, and the configurations for these are dicussed in later 
sections.
